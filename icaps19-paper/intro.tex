\section{Introduction}


% THE CONTENTS BELOW IS IRRELEVANT (CENTROID STUFF, NOT ADDRESSED IN THIS PAPER)
%In this document we introduce a new type of a redundant symbol, called a \emph{centroid redundant symbol}.
%In preliminary discussions in Delft we have called it the $\gamma$ symbol. 
%(We need to choose a final name.)

\ab{My comments made across the paper ignore feedback that I have already given to Mattia. He will integrate that feedback into the paper.}

\ab{Drafted a few introductory paragraphs below. Please review and change as you see fit.}

Path planning has been an important AI problem for decades, with applications in domains such as games and robotics. For example, in the past, path planning used to be the only AI implemented in video games. Despite significant improvements shown in the literature, path planning on gridmaps remains an active area of research. This is demonstrated, for instance, by the interest shown in the Grid-based Path Planning Competition GPPC~\cite{DBLP:conf/socs/SturtevantTTUKS15}.

Compressed Path Databases (CPDs) represent a state-of-the-art approach, in terms of speed, to optimal pathfinding on gridmaps~\cite{DBLP:conf/socs/SturtevantTTUKS15,DBLP:conf/aips/SalvettiBGHS18}. A CPD is a precomputed datastructure that provides an optimal move from any node $s$ towards any node $t$. The data is precomputed with a series of independent Dijkstra searches, one from each node $s$ on the graph (gridmap). The Dijkstra algorithm is slightly modified to output a set of optimal moves, as opposed to distances. More specifically, this provides one optimal move from the source at hand $s$ towards any other node $t$ on the graph. The set of optimal moves is compressed. CPDs can provide full optimal paths fast, by repeatedly looking up the next optimal move to make, until the target is achieved. Furthermore, a CPD provides fast any prefix of an optimal path. This is important to reduce the so-called first-move lag, when an agent needs to wait until it knows in which direction to move. Many other pathfinding techniques, including search-based methods, such as A*, know the first optimal move only when they know the entire solution. In contrast, a CPD provides the first move much faster, independently of the rest of the path.


Recent work has significantly improve the size of CPDs~\cite{strasser-et-al-2014,DBLP:conf/aips/SalvettiBSG17}. However, on large maps, the size of a CPD can represent a bottleneck.

In this work, we focus on new compression ideas.

\ab{I need to run now, but I volunteer to finish the introduction.}

%In the past we have considered the idea that a given move in a CPD can represented with two redundant symbols.
%This way, at compression time, we can choose the symbol that leads to a better compression.
%Specifically, we have considered move symbols (N, S, W, E, SE, SW, NE, NW) and a special symbol for default moves.
%This is the work that Matteo is performing at the moment.
