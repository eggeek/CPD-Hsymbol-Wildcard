\begin{abstract}
Compressed Path Databases (CPDs) are a state-of-the-art method for
path planning. They record, for each start position, 
an optimal first move to reach any target position.  
Computing an optimal path with CPDs 
%involves a recursive sequence of lookups into the database, 
%each time extracting and then following a first move. 
%This procedure 
is extremely fast and requires no state-space search.
The main disadvantages are overhead related:
building a CPD usually involves an all-pairs precomputation, and 
storing the result incurs often prohibitive space overheads.
%When the CPD is too large an obvious approach
%is to store only information for some targets while maintaining bounded
%suboptimal path finding.  Somewhat surprisingly this simple approach
%fails. 
Previous research has focused on reducing the size of CPDs and/or 
improving their online performance.  
In this paper we consider a new type of CPD, which can also dramatically 
reduce preprocessing times.
Our idea involves computing first-move data for only selected target
nodes; chosen in such a way as to guarantee that the cost of any extracted 
path is within a fixed bound of the optimal solution.
In a range of empirical results we show that our new bounded suboptimal
CPDs improve preprocessing times, reduce storage costs,
and compute paths more quickly -- all in exchange for only a small
amount of suboptimality.
\end{abstract}
